\documentclass[11pt]{llncs}
\usepackage[scale=.78]{geometry}

\usepackage[backref]{hyperref}
\usepackage{amsmath,amsfonts,amssymb}

\newcommand\ignore[1]{}

\begin{document}

\title{Boolean/Arithmetic Conversions  for any Order}

\author{
Jean-S\'ebastien Coron}

\institute{~}

\maketitle

\begin{abstract}
At CHES 2001 Goubin presented a nice method for converting from boolean
masking to arithmetic masking and conversely. However the method is
only secure against first-order attacks. In this paper we provide a
generalization to any order.
\end{abstract}

\section{Introduction}

Let $x$ be a sensitive variable. 

Boolean masking: $x=x' \oplus r$.

Arithmetic masking: $x=A+ r \bmod 2^k$, where $k$ is the register
size.


\section{From Boolean to Arithmetic Masking}

It is shown in \cite{goubin} that the following function is affine
with respect to the Boolean operation:
$$ \Psi_{x'}(r) = (x' \oplus r) -r \mod 2^k$$

\begin{theorem}[Goubin \cite{goubin}]
The function $$ \Psi_{x'}(r) = (x' \oplus r) -r \mod 2^k$$ is affine
over $GF(2)$.
\end{theorem}

\begin{proof}
The proof in \cite{goubin} is based on the following equality.
$$ \Psi_{x'}(r)=x' \oplus \bigoplus\limits_{i=1}^{k-1} \left( \left(
\bigwedge\limits_{j=1}^{i-1} 2^j \bar{x'} \right) \wedge (2^i x') \wedge
(2^i r) \right)$$
This equality is proved by induction in \cite{goubin}; the affine
property of $\Psi_{x'}(r)$ follows immediatly.  However, it is
difficult to understand intuitively why such equality holds. In this
section we provide a hopefully more enlightening  proof of the affine
property of $\Psi_{x'}(r)$. To do.
\qed
 \end{proof}

The conversion from boolean to arithmetic masking is then
straightforward. Given $x',r$ such that 
$x=x' \oplus r$
we must compute $A$ such that $x=A+r$, which gives:
$$ A=(x' \oplus r)-r=\Psi_{x'}(r)=\Psi_{x'}(r \oplus r_2) \oplus
\left(\Psi_{x'}(r_2) \oplus \Psi_{x'}(0) \right)$$
for a random $r_2$. The technique is clearly secure against
first-order attacks.


\subsection{Generalization to any Order}

\label{s:firstgen}

In this section we describe a first generalization to any order based on the
previous Goubin 
method; however it does not seem to work. 

We want to convert from:
$$ x=x_1 \oplus x_2 \oplus \ldots \oplus x_n$$
to
$$x =A_1 + A_2 + \ldots + A_n$$
without any $(n-1)$-th order leakage.

We proceed iteratively by computing:
\begin{eqnarray*}
x& =& x_1 \oplus x_2 \oplus \ldots \oplus x_n \\
x & = & A_1 + (x_2 \oplus \ldots \oplus x_n) \\
x & = & A_1 + A_2 + (x_3 \oplus \ldots \oplus x_n) \\
& \vdots & \\
x & = & A_1 + \ldots + A_{n-2} + (x_{n-1} \oplus x_n) \\
x & = & A_1 + \ldots + A_{n-2}+ A_{n-1} + A_n
\end{eqnarray*}

In the following we focus on the first conversion; the remaining
conversions proceed similarly.
\begin{eqnarray*}
x& =& x_1 \oplus x_2 \oplus \ldots \oplus x_n \\
x & = & A_1 + (x_2 \oplus \ldots \oplus x_n) 
\end{eqnarray*}
which gives:
\begin{eqnarray*}
A_1 & = & x_1 \oplus (x_2 \oplus \ldots \oplus x_n) - (x_2 \oplus \ldots
\oplus x_n) \\
A_1 & = & \Psi_{x_1}(x_2 \oplus \ldots \oplus x_n) \\
A_1 & = & \Psi_{x_1}(x_2) \oplus \Psi_{x_1}(x_3) \ldots \oplus
\Psi_{x_1}(x_n) \oplus u
\end{eqnarray*}
where $u=\Psi_{x_1}(0)=x_1$ if $n$ is odd, and $u=0$ otherwise. This
is thanks to the affine property of $\Psi_{x_1}(r)$.

Now one could compute the $\Psi_{x_1}(x_i)$ for $2 \leq i \leq n$
separately and eventually xor them, but the xor operation could leak
information on $x$ (even though the result $A_1$ does not leak information
on $x$). One could try to randomize the process by xoring the
intermediate variables by a sequence of random $r_i$'s, but that does
not seem to work. 

The complexity of the first step is ${\cal O}(n)$; therefore the total
complexity is ${\cal O}(n^2)$. 

\section{From Boolean to Arithmetic Masking for any Order}

\label{s:boolarithany}

In this section we describe a conversion method from Boolean to
Arithmetic masking that seems to work for any order. However it is
less efficient, with complexity ${\cal O}(n^3)$.

We assume that there is an index $j$ such that the senstitive variable
$x$ is shared as follows.
$$ x=A_1 + A_2 + \ldots+ A_j + (x_{j+1} \oplus \ldots \oplus x_n)$$
Initially we have $j=0$ (Boolean masking), and eventually $j=n-1$
(arithmetic masking).

We show how to go from step $j$ to step $j+1$, without any small order
leakage. The technique consists in writing:

\begin{eqnarray*}
x & = & A_1 + \ldots +A_j + (x_{j+1} \oplus \ldots \oplus x_n)\\
x & = & A_1 + \ldots +A_j + (-r)+ \left( r + (x_{j+1} \oplus \ldots
\oplus x_n) \right) \\
x & = & A_1 + \ldots + A_{j} + (-r)+  \left( (r_{j+1} \oplus \ldots \oplus
r_n) + (x_{j+1} \oplus \ldots
\oplus x_n)   \right) \\
x & = & A_1 + \ldots + A_{j} + A_{j+1}+  \left( x'_{j+2} \oplus \ldots \oplus
x'_n \right)
\end{eqnarray*}
where $r_{j+1} \oplus \ldots \oplus r_n=r$, and $A_{j+1}=(-r)$.

Therefore one must perform the addition:
$$(r_{j+1} \oplus \ldots \oplus
r_n) + (x_{j+1} \oplus \ldots
\oplus x_n)=  x'_{j+2} \oplus \ldots \oplus
x'_n $$
without any low-order leakage. This corresponds to a secure evaluation
of the
circuit for addition, with shares of size $n-j= {\cal O}(n)$. For
$k$-bit registers, the addition circuit has size ${\cal
  O}(k)$. Therefore using the ISW technique  \cite{isw} this can be done in time ${\cal O}(k \cdot
n^2)$. Therefore the full complexity is ${\cal O}(k \cdot n^3)$.

Note that the technique does not use Goubin's conversion
method. Goubin's conversion method could be used in the last step, to
convert $x_{n-1} \oplus x_n$ into $A_{n-1}+ A_n$, but this would not
modify the total complexity. 

\section{From Arithmetic to Boolean Masking}

Goubin also described in \cite{goubin} a technique for converting from
arithmetic to boolean masking, secure against first-order leakage. However it
is more complex than from boolean to arithmetic masking. Its
complexity is ${\cal O}(k)$ for registers of $k$ bits.
It is based on the following theorem.

\begin{theorem}[Goubin \cite{goubin}]
\label{t:goubin2}
If we denote $x'=(A+r) \oplus r$, we also have $x'=A \oplus u_{k-1}$,
where $u_{k-1}$ is obtained from the following recursion formula:
$$\left\{
\begin{array}{l}
u_0=0 \\
\forall k \geq 0, u_{k+1}=2 [u_k \wedge (A \oplus r) \oplus (A \wedge
  r) ]
\end{array}
\right.
$$
\end{theorem}


\section{From Arithmetic to Boolean Masking for any Order}

\label{s:arithboolany}

We use the same technique as in Section \ref{s:boolarithany}. We write:

\begin{eqnarray*}
x & = & A_1 + \ldots +A_{j-1}+ A_j + (x_{j+1} \oplus \ldots \oplus x_n)\\
x & = & A_1 + \ldots +A_{j-1} + \left( A_j + (x_{j+1} \oplus \ldots
\oplus x_n) \right) \\
x & = & A_1 + \ldots +A_{j-1} + \left( (r_{j} \oplus \ldots \oplus
r_n) + (x_{j+1} \oplus \ldots
\oplus x_n) \right) \\
x & = & A_1 + \ldots + A_{j-1} +  \left( x'_{j} \oplus \ldots \oplus
x'_n \right)
\end{eqnarray*}
where $r_{j} \oplus \ldots \oplus r_n=A_j$. The complexity for one
step is ${\cal O}(k \cdot n^2)$ and the full complexity is again
${\cal O}(k \cdot n^3)$. 

\section{From Boolean to Arithmetic Masking for any Order}

We show a variant of the technique from Section \ref{s:firstgen} that
could work.

We use a procedure ${\sf PartialConvert}$ which is defined as follows:
$${\sf PartialConvert}(x_1,\ldots,x_d)=(y_1,\ldots,y_{d-1})$$
where
$$ (y_1 \oplus \ldots \oplus y_{d-1}) + (x_2 \oplus \ldots \oplus
x_d)=x_1 \oplus \ldots \oplus x_d$$
In other words, {\sf PartialConvert} provides a $(d-1)$-th boolean sharing of
$A_1$, where 
$$A_1 +  (x_2 \oplus \ldots \oplus
x_d)=x_1 \oplus \ldots \oplus x_d$$


Using Goubin's convertion method it is easy to obtain such procedure
${\sf PartialConvert}$:
\begin{eqnarray*}
A_1 & = & x_1 \oplus (x_2 \oplus \ldots \oplus x_d) - (x_2 \oplus \ldots
\oplus x_d) \\
A_1 & = & \Psi_{x_1}(x_2 \oplus \ldots \oplus x_d) \\
A_1 & = & \Psi_{x_1}(x_2) \oplus \Psi_{x_1}(x_3) \ldots \oplus
(\Psi_{x_1}(x_d) \oplus u) \\
A_1 & = & y_1 \oplus \ldots \oplus y_{d-1}
\end{eqnarray*}
where $u=\Psi_{x_1}(0)=x_1$ if $n$ is odd, and $u=0$ otherwise. This
is thanks to the affine property of $\Psi_{x_1}(r)$. We take
$y_i=\psi_{x_1}(x_{i+1})$ for $2 \leq i \leq d$. It is probably a good
idea to further re-randomize the $y_i$'s. The complexity of ${\sf
  PartialConvert}$ is ${\cal O}(d)$.

Using {\sf PartialConvert} with input a $d$-th order Boolean
sharing, we obtain as output a arithmetic sum of $2$ Boolean sharing
of order $d-1$. We can therefore apply the {\sf PartialConvert}
procedure recursively on both parts.

The total complexity to convert a $n$-th order Boolean masking into
arithmetic masking is then
$\tilde{ \cal O}(2^n)$. This is worse than the ${\cal O}(n^3)$
complexity from
Section \ref{s:boolarithany} but this might be more advantageous for
small $n$, since we don't have to perform bit manipulations. More precisely
taking into account the register size, the complexity is $\tilde{ \cal
  O}(2^n)$ instead of ${\cal O}(k \cdot n^3)$ in Section
\ref{s:boolarithany}.  So this method is more
advantageous for small $n$ and large register size $k$; this may
correspond to what is used in practice.

Finally we note that we could use a similar technique for the
convertion from arithmetic to Boolean masking. The complexity would
be  $\tilde{ \cal
  O}(2^n)+ {\cal O}(k \cdot n^2)$ instead of ${\cal O}(k \cdot n^3)$ in Section
\ref{s:arithboolany}

\section{Yet Another Variant}

We note that in both Sections \ref{s:boolarithany} and
\ref{s:arithboolany} we must compute:
$$ r+ (x_{j+1} \oplus  \cdots \oplus x_n)=x'_{j} \oplus \cdots
\oplus x'_n$$
Now instead of using a circuit of size ${\cal O}(k \cdot n^2)$ for
performing the arithmetic addition, we 
 can actually use Goubin's second method for conversion from
 arithmetic too boolean masking.

Namely denoting $x'=x'_{j} \oplus \cdots
\oplus x'_n$, we have:
$$x' \oplus r=x'_{j} \oplus \cdots
\oplus (x'_n \oplus r)=(A+r) \oplus r$$
where 
$$A=x_{j+1} \oplus  \cdots \oplus x_n$$
Therefore we use the compute the  recurrence from Theorem
\ref{t:goubin2}, with $x'=(A+r) \oplus r=A \oplus u_{k-1}$, where:
$$\left\{
\begin{array}{l}
u_0=0 \\
\forall k \geq 0, u_{k+1}=2 [u_k \wedge (A \oplus r) \oplus (A \wedge
  r) ]
\end{array}
\right.
$$
where we compute $u_k$ and $A$ with $n$ boolean shares. Therefore the
result $x' =A \oplus u_{k-1}$ has also $n$ boolean shares (if there
are too many boolean shares, we can simply reduce the number of
shares). 

The complexity of this step is still ${\cal O}(k \cdot n^2)$
operations, and the total complexity is still ${\cal O}(k \cdot
n^3)$. However, this latter method might be easier to implement.


\begin{thebibliography}{30}
\bibitem{goubin}
L. Goubin, {\sl A Sound Method for Switching between Boolean and
  Arithmetic Masking}. Proceedings of CHES 2001.

\bibitem{isw}
Y. Ishai, A. Sahai and D. Wagner, 
{Private Circuits: Securing Hardware against Probing Attacks},
Proceedings of Crypto 2003.

\end{thebibliography}

\end{document}
